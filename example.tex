\documentclass[a4paper, 12pt, english]{article}
\usepackage[legalpaper, portrait, margin=1in]{geometry}
\usepackage[T1]{fontenc}
\usepackage[utf8]{inputenc}
\usepackage{titling}
\usepackage{hyperref}

% here we import the package
\usepackage{./lingproblems}

% remove page numbering
\pagenumbering{gobble}%

\author{}
\date{}
% we reduce the space between the title and the beginning of the document
\title{\textbf{My favourite problems}\vspace{-4em}}


% we reduce the space between the top of the page and the title
\setlength{\droptitle}{-7em}

\begin{document}

\maketitle{}

% here we add a new problem
\begin{problem*}{Michał Boroń}

Below you can find some equations and inequalities written in the names of Basque and reconstructed Iberian numbers.
Linguists were astonished to encounter some striking similarities between these two languages.
In the following the Basque names are mixed together with the Iberian reconstructions.\\
The spelling of the Iberian names is based on Eduardo Ordu\~na's research.[citation]
\begin{align}
o\acute{r}keiba\acute{r}ban &= hogeita\; lau + sisbi\label{eq:first} \\
aba\acute{r}\acute{s}ei &= lau^{bi} \\
hamabi + laur &= hamasei \\
hogei &= irur \times sisbi - bat \\
hiru \times irur + ban &= bi \times borste \\
laur \times bost &= o\acute{r}kei \\
o\acute{r}kei - hamar &= hiru + zazpi \\
aba\acute{r}kebi + hiru &= hamabost \\
borste &< zazpi\label{eq:last}
\end{align}

% here we define tasks
\begin{tasks}
\task{Rewrite the equations from \ref{eq:first} to \ref{eq:last} in numerals.}
\task{Match in pairs the corresponding names of numbers from the set $\{1,2, ..., 7, 10, 20\}$.
One number has just one form for both languages.}
\end{tasks}

% here we give some info about the language and possibly about some atypical phonemes used in the problem
\begin{langinfo}
Iberian belonged to an unclassified language family, was spoken on the Mediterranean coast of the Iberian Peninsula
and became extinct around 1st--2nd century AD.\\
\indent Basque is considered to be a language isolate.
It is spoken in the Basque Country in the north of Spain and the south of France by approx. 750 000 native speakers (2016).\\

Note that \textbf{\'r} and \textbf{r} as well as \textbf{\'s} and \textbf{s} should be treated as two distinct consonants.
\end{langinfo}

% we end the problem
\end{problem*}



% here we add a new problem whose name isn't going to be shown
\begin{problem}{Another problem}{}

This very hard problem does have a name. Yet, it doesn't have an author.

% here we define tasks
\begin{tasks}
\task{This is the first task.}
\end{tasks}

% here we give some info about the language and possibly about some atypical phonemes used in the problem
\begin{langinfo*}
This is my favourite language spoken in X by Y speakers.
\end{langinfo*}

% we end the problem
\end{problem}


\end{document}