\documentclass[a4paper, 12pt]{article}
\usepackage[portrait, margin=1in]{geometry}
\usepackage[T1]{fontenc}
\usepackage[utf8]{inputenc}
\usepackage{titling}

% here we import the package
\usepackage{./lingproblems}

% removes page numbering
\pagenumbering{gobble}%

\author{}
\date{}
% we reduce the space between the title and the beginning of the document
\title{\textbf{My favourite problems}\vspace{-4em}}

% we reduce the space between the top of the page and the title
\setlength{\droptitle}{-7em}

\begin{document}

\maketitle{}

% here we add a new problem which doesn't have a title
\begin{lingproblem*}{Michał Boroń}

Below you can find some expressions written in Basque and in reconstructed Iberian numbers.
Linguists were astonished to encounter some striking similarities between these two languages.
In the following, the Basque names are mixed together with the Iberian reconstructions.
\begin{align}
o\acute{r}keiba\acute{r}ban &= hogeita\; lau + sisbi\label{eq:first} \\
aba\acute{r}\acute{s}ei &= lau^{bi} \\
hamabi + laur &= hamasei \\
hogei &= irur \times sisbi - bat \\
hiru \times irur + ban &= bi \times borste \\
laur \times bost &= o\acute{r}kei \\
o\acute{r}kei - hamar &= hiru + zazpi \\
aba\acute{r}kebi + hiru &= hamabost \\
borste &< zazpi\label{eq:last}
\end{align}

% here we define tasks
\begin{tasks}
% here is a task without any points indicated
\task{Rewrite the expressions from \ref{eq:first} to \ref{eq:last} in numerals.}

% here is a task with a given number of points
\taskp{Find the corresponding names in both languages for the numbers from the following set 
$\{1,2, ..., 7, 10, 20\}$.
One number has just one form in both languages.}{9 points}
\end{tasks}

% here we give some info about the language and possibly about some atypical phonemes used in the problem
\begin{langinfo}
Iberian belonged to an unclassified language family and
was spoken on the Mediterranean coast of the Iberian Peninsula.
It became extinct around 1st--2nd century AD.

Basque is considered to be a language isolate.
It is spoken in the Basque Country in the north of Spain and the south of France by approx. 750 000 native speakers (2016).

Note that \textbf{\'r} and \textbf{r} as well as \textbf{\'s} and \textbf{s} should be treated as two distinct consonants.
\end{langinfo}

% we end the problem
\end{lingproblem*}



% here we add a new problem whose name is going to be shown
\begin{lingproblem}{The Hardest Problem}

% the problem's description
This very hard problem does have a name. Yet, it doesn't have an author.

% here we define tasks
\begin{tasks}
\task{This is the first task with no points indicated.}
\end{tasks}

% here we give some info about the language and possibly about some atypical phonemes used in the problem
% due to usage of langinfo*, no author is shown
\begin{langinfo*}
This is my favourite language spoken in X by Y speakers.
\end{langinfo*}

% we end the problem
\end{lingproblem}


\end{document}